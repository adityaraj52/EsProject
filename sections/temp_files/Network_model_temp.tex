\chapter{Network Model}\label{ch:model}


In our application scenario, the \chs disseminate processing ability/interests into the network using short beacons. Intended senders try to access channel using the mac-protocol. On grant of access to the channel, they send the data. This data is compressed and encrypted using low end compression and encryption algorithm respectively. Further, using a suitable routing protocol the data is routed to \chs. They finally report the data to BS. Therefore, this research also includes realizing different aspects of a \acf{WSN} like Data Collection, Target tracking, Routing, Data Aggregation, Data Dissemination and Synchronization. 

It can be inferred from the above description that:
\begin{itemize}
	
	\item All sensors are not homogeneous. They have different processing power and communication capabilities.
	
	\item Sensor nodes sense the environment and send data to nearby(1 or 2 hop neighbours) Cluster Heads, which are the special \acp{SN} with high processing and energy capabilities, periodically.
	
\end{itemize}

Because of huge similarity between our model and the cluster-based model, we decide to represent our scenario in cluster-based fashion instead of Flat, Tree-Based or Chain-Based. So in our model, Higher processing power nodes act as Cluster Heads inside \acf{WSN} and they act as data collectors and processing centers.

In this context, we will also address following issues in clustering:

\begin{itemize}
\item Clustering Formation: Partitioning a network in clusters.

\item Cluster head election: This can be done in three different ways:

\begin{itemize}
\item Similar to LEACH protocol: if Probability>threshold make leader, rest become followers

\item By Election: All broadcast their infos and thus from the KB, leader is chosen

\item By Base Station: need lot of communications with Base Station, so seldom used

\end{itemize}
\item set up phase and data transmission phase
\item Inter-cluster and Intra-cluster collision resolution.

\end{itemize}

\noindent Following assumptions are made in modeling the network:

\begin{itemize}
	\item Radio Channel is symmetric in terms of Energy transmission between A and B.
	
	\item Sensors have limited sensing and sending range.
	
	\item Nodes are not mobile.
	
	\item Sensors are uniformly dispersed within a square field A
	
	\item Sensors and Base-Station are stationary after deployment
	
	\item The communication is based on the single-hop
	
	\item Communication is symmetric and a sensor can compute the approximate distance based on the received signal strength if the transmission power is given
	
	\item AIl sensors are location-unaware
\end{itemize}

\noindent The tasks involved for creating such a model can be summarized in following key points:

\begin{itemize}
	\item Time Synchronisation for SN using Beacons by ClusterHeads to SensorNodes and Other ClusterHeads. Assuming that this exists??
	
	\item At Physical layer: Implementing B-MAC to arbitrate channel access to SN.
	
	\item At Network layer: Using a suitable Routing protocol for ClusterBased Heterogeneous \acf{WSN} to forward sensed data to ClusterHeads.
	
	\item At application level: Implementation of low end Compression and encryption and aggregation of data collected at ClusterHeads.
	
	\item Testing the whole set up in a simulation environment
	
	\item Testing the model on actual Physical environment.
	
\end{itemize} 




