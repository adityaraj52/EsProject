% !TEX root = ./TUCthesis.tex

%% ****************************************************************************************************
%% 0. Set the encoding of your files. UTF-8 is the only sensible encoding nowadays. If you can't read
%% äöüßáéçèê∂åëæƒÏ€ then change the encoding setting in your editor, not the line below. If your editor
%% does not support utf8 use another editor!
%% ****************************************************************************************************
\PassOptionsToPackage{utf8}{inputenc}
\usepackage{inputenc}
\PassOptionsToPackage{T1}{fontenc} % T2A for cyrillic
\usepackage{fontenc}   
\usepackage{adjustbox}  


%% ****************************************************************************************************
%% 1. Configure classicthesis for your needs here. Frequently considered changes:
%% - Remove "drafting" below in order to deactivate the time-stamp on the pages
%% - Use "floatperchapter" if many floats (figures, tables, equations) exist
%% - Choose "dottedtoc" if you like page numbers to be right-aligned in the table of contents
%% For more available options see the source code of classicthesis.sty 
%% ****************************************************************************************************
\PassOptionsToPackage{	eulerchapternumbers,% use big chapter numbers in Euler math font
	listings,% add support for code listings using \lstlisting
	pdfspacing,% use pdftex for improved letter spacing
	dottedtoc,% use dotted lines in table of contents (right-align page numbers)
	floatperchapter,% enumerate tables/figures per chapter instead of consecutively
	%drafting,% add version stamp at the bottom of each page
	subfig,% allow for the inclusion of subfigures in any figure environment
	beramono,% use Beramono font for monospaced text
	eulermath% use Euler math font
}{classicthesis}                                   


%% ****************************************************************************************************
%% 2. Personal thesis-related data to auto-generate title page etc
%% Make sure to terminate each of these fields with \xspace for proper formatting
%% ****************************************************************************************************
\newcommand{\myTitle}{Location-Aware Data Processing in Heterogeneous Wireless Sensor Networks\xspace}
\newcommand{\myName}{Aditya Raj\xspace}
\newcommand{\myID}{381678\xspace}
\newcommand{\myThesisType}{ITIS Resarch Project Report\xspace}
\newcommand{\mySubmitDate}{15 April 2017\xspace} % format your date according to your thesis language!
\newcommand{\myThesisNumber}{ES-P005\xspace} % ask supervisor for this number - do not leave at 000!

\newcommand{\myProf}{Dr.-Ing.~Andreas Reinhardt\xspace} % Advisor and first referee
\newcommand{\myOtherProf}{Prof.~Dr.~Sven Hartmann\xspace} % Second referee
\newcommand{\myDepartment}{Embedded Systems\xspace} % Group within the Department of Informatics
\newcommand{\myLocation}{Clausthal-Zellerfeld\xspace} % Location where you will sign the statutory declaration
\newcommand{\myUni}{Technische Universit\"at Clausthal\xspace}

%% ****************************************************************************************************
%% 3. TU Clausthal specifics
%% Note that TUC only specifies four colors (one of which is a very light gray, thus unusable)
%% If you need to color text, either use \textcolor{TUDred}{text to color}
%% Alternatively, if the color shall be applied to links, specify it in the hyperref config below
%% ****************************************************************************************************
\usepackage[dvipsnames]{xcolor} % allow for TUC colors
\definecolor{TUCgreen}{rgb}{0,0.55,0.31}
\definecolor{TUCgrey1}{rgb}{0.5,0.5,0.5}
\definecolor{TUCgrey2}{rgb}{0.9,0.9,0.9}
\definecolor{TUCred}{rgb}{0.55,0.11,0}
\usepackage{tikz} % allow us to place the TUC logo on front page
\usepackage{translations} % allow us to automatically use english front matter

\usepackage{epstopdf} % for the TUC logo on the front page, when using pdflatex on Windows
\epstopdfsetup{update}

%% uncomment one of the following options if your thesis must be typeset in sans-serif font
%\usepackage[default,osfigures]{opensans}% use a sans-serif font
%\usepackage[sfdefault]{FiraSans}% use a sans-serif font


%% ****************************************************************************************************
%% 4. Loading some handy packages
%% ****************************************************************************************************
\PassOptionsToPackage{ngerman,american}{babel}   % add language support
\usepackage{babel}%	allow for proper hyphenation in languages defined above
\usepackage{csquotes}%	set quotation marks according to used language

\PassOptionsToPackage{%	add bibliography support
	backend=bibtex8,%		use BibTeX (change to biber if you know what you are doing)
	bibencoding=ascii,%		assume that BibTeX file is saved as ASCII
	language=auto,%		use hyphenation={} field in BibTeX to apply correct hyphenation
	style=numeric-comp,%	use numeric format & compress 1,2,3 to 1-3 (alternatives: numeric,alphabetic)
	sorting=nyt,% 			sorting by name, year, title
	maxbibnames=10, %	how many authors are listed before adding "et al."?
	%backref=true,%			show on which page a reference has been cited
}{biblatex}
\usepackage{biblatex}
\usepackage{algorithm}
\usepackage[noend]{algpseudocode}
    

%% math environments and more by the American Mathematics Society 
\PassOptionsToPackage{fleqn}{amsmath}
\usepackage{amsmath}

%% allow the use of acronyms (that automatically show up in the list of acronyms)
%% define any acronyms in FrontBackmatter/Contents.tex
%% Then use the acronyms in the text as follow:
%% - \ac{UML}		full spelled-out on first use, subsequently abbreviated
%% - \acp{UML}	plural case of the aforementioned one
%% - \acf{UML}	full spelled-out version, even if used before
\PassOptionsToPackage{printonlyused,smaller}{acronym} 
\usepackage{acronym} % nice macros for handling all acronyms in the thesis

\usepackage{textcomp} % fix warning with missing font shapes
\usepackage{scrhack} % fix warnings when using KOMA with listings package          
\usepackage{xspace} % to get the spacing after macros right  
\usepackage{mparhack} % get marginpar right
\usepackage{fixltx2e} % fixes some LaTeX stuff --> since 2015 in the LaTeX kernel (see below)
\usepackage{lipsum} % generate dummy text to see layout

%% ****************************************************************************************************
%% 5. Setup floats: tables, (sub)figures, and captions
%% ****************************************************************************************************

%% define how floats should be aligned. Make sure to use this in any float you add!
\newcommand{\myfloatalign}{\centering}

%% better tables. Make sure to use \tableheadline{TEXT} in each field of the headline row.
\usepackage{tabularx}
\setlength{\extrarowheight}{3pt} % increase table row height
\newcommand{\tableheadline}[1]{\multicolumn{1}{c}{\spacedlowsmallcaps{#1}}}

%% formatting for float captions (smaller font size)
\usepackage{caption}
\captionsetup{font=small}
\usepackage{subfig}  


%% ****************************************************************************************************
%% 6. Prepare code listings to use monospaced font and coloring
%% ****************************************************************************************************
\usepackage{listings} 
\lstset{language=[LaTeX]Tex,%C++,
	morekeywords={PassOptionsToPackage,selectlanguage},
	basicstyle=\small\ttfamily,% code formatting
	stringstyle=\rmfamily,% comment formatting
	keywordstyle=\color{RoyalBlue},% color of operators
	%identifierstyle=\color{NavyBlue},% color of variables
	commentstyle=\color{TUCgrey1}\ttfamily,
	numbers=left,% set this to "left" or "none" if line numbers are needed or not
	numberstyle=\scriptsize,
	numbersep=8pt,
	showstringspaces=false,% set this to "true" or "false" if spaces shall be visualized
	breaklines=true,% break too long lines
	belowcaptionskip=.75\baselineskip
} 


%% ****************************************************************************************************
%% 7. Clickable references in PDF file and citation back-references
%% ****************************************************************************************************
\PassOptionsToPackage{pdftex,hyperfootnotes=true,pdfpagelabels}{hyperref}
\usepackage{hyperref}
\pdfcompresslevel=9
\pdfadjustspacing=1 
\PassOptionsToPackage{pdftex}{graphicx}
\usepackage{graphicx} 

%% Reference coloring 
\hypersetup{%
	%draft,% no hyperlinking at all (useful in b/w printouts, but no clickable references)
	colorlinks=true, linktocpage=true,%	if you want hyperlinks, use this line for colored ones - OR
	%colorlinks=false, linktocpage=false, pdfborder={0 0 0},% this line for black links (e.g., for printing)
	pdfstartpage=3, pdfstartview=FitV, hypertexnames=true, pdfhighlight=/O,%
	breaklinks=true, pdfpagemode=UseNone, pageanchor=true, pdfpagemode=UseOutlines,%
	plainpages=false, bookmarksnumbered, bookmarksopen=true, bookmarksopenlevel=1,%
	urlcolor=TUCred, linkcolor=TUCgreen, citecolor=TUCgreen,%
	%urlcolor=Black, linkcolor=Black, citecolor=Black, pagecolor=Black,% any dvipscolors name can be used
	pdftitle={\myTitle},%
	pdfauthor={\textcopyright\ \myName, \myUni},%
	pdfsubject={},%
	pdfkeywords={},%
	pdfcreator={pdfLaTeX},%
	pdfproducer={LaTeX with hyperref and classicthesis}%
}   

%% Setup autoreferences
\makeatletter
\@ifpackageloaded{babel}{%
	\addto\extrasamerican{%
		\renewcommand*{\figureautorefname}{Figure}%
		\renewcommand*{\tableautorefname}{Table}%
		\renewcommand*{\partautorefname}{Part}%
		\renewcommand*{\chapterautorefname}{Chapter}%
		\renewcommand*{\sectionautorefname}{Section}%
		\renewcommand*{\subsectionautorefname}{Section}%
		\renewcommand*{\subsubsectionautorefname}{Section}%     
	}%
	\addto\extrasngerman{% 
		\renewcommand*{\paragraphautorefname}{Absatz}%
		\renewcommand*{\subparagraphautorefname}{Unterabsatz}%
		\renewcommand*{\footnoteautorefname}{Fu\"snote}%
		\renewcommand*{\FancyVerbLineautorefname}{Zeile}%
		\renewcommand*{\theoremautorefname}{Theorem}%
		\renewcommand*{\appendixautorefname}{Anhang}%
		\renewcommand*{\equationautorefname}{Gleichung}%        
		\renewcommand*{\itemautorefname}{Punkt}%
	}%  
	\providecommand{\subfigureautorefname}{\figureautorefname}%             
}{\relax}
\makeatother


%% ********************************************************************
%% 8. Last, but not least...
%% ********************************************************************
\defbibheading{classicthesis}[\bibname]{%
	\cleardoublepage\phantomsection
	\manualmark
	\markboth{\spacedlowsmallcaps{#1}}{\spacedlowsmallcaps{#1}}%
	\addtocontents{toc}{\protect\vspace{\beforebibskip}}%
	\addcontentsline{toc}{chapter}{\tocEntry{#1}}%
	\chapter*{#1}\label{app:bibliography}%
}
\usepackage{classicthesis} 
\ifthenelse{\boolean{@drafting}}{\renewcommand{\PrelimText}{\footnotesize[draft created on \,\today\ at \thistime\xspace (\scriptsize delete \texttt{drafting} option in \texttt{thesis-config.tex} to remove this line) \footnotesize]}}{}


%% ****************************************************************************************************
%% 9. Further adjustments (talk to supervisor before changing anything below)
%% ****************************************************************************************************

\ifthenelse{\boolean{@drafting}}{
	\usepackage[textsize=footnotesize]{todonotes} % allow for inserting \todo{..} items
}{
	\usepackage[disable]{todonotes} % allow for inserting \todo{..} items
}

%% increase line spacing so your supervisor can add remarks more easily
%\linespread{1.5} 

%% add this if paragraphs should not be indented (should never be the case, though)
%\parindent0pt 

\let\marginpar\oldmarginpar %revert classicthesis change to make todonotes work
\addtolength{\marginparwidth}{10mm} % make more space for todo notes
\addtolength{\textheight}{-12mm}\addtolength{\topmargin}{14mm} % increase top margin

%% DEBUG use only
%\usepackage{showframe}

%My packages

\usepackage{listings}
\usepackage{lipsum}
\usepackage{parcolumns}

